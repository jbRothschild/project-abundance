\documentclass[9pt,twoside,lineno]{pnas-new}
% Use the lineno option to display guide line numbers if required.

\templatetype{pnassupportinginfo}

\title{Your main manuscript title}
\author{Author1, Author2 and Author3 (complete author list)}
\correspondingauthor{Corresponding Author name.\\E-mail: author.two@email.com}

\begin{document}

%% Comment out or remove this line before generating final copy for submission; this will also remove the warning re: "Consecutive odd pages found".
\instructionspage  

\maketitle

%% Adds the main heading for the SI text. Comment out this line if you do not have any supporting information text.
\SItext
In the following we provide additional derivations and discussion as a complimentary to the main text.
\section*{Approximations of Species Abundance Distribution }

\subsection*{Derivation of zero flux in $n_i$ - Global balance equation - derivation for the exact SAD}
\label{App:ZeroFlux}


Consider the multi-dimensional  master equation 
\begin{eqnarray}
    \partial_tP(n_1,n_2\dots,n_S) &=& \sum_{i}\left\{ q_{i}^+(\vec{n}-\vec{e_i})P(\vec{n}-\vec{e_i})+q^-_{i}(\vec{n_i}+\vec{e_i}) P(\vec{n}+\vec{e_i})-\left[q_{i}^+(\vec{n})+q_{i}^-(\vec{n}) \right]P(\vec{n}) \right\}
\end{eqnarray}
 where $q^+_{n_i}(\vec{n})$ and $q^-_{n_i}(\vec{n)}$ represents the birth and death rate of species $i$ (respectively), which are generally depends in $\vec{n}=(n_1,\dots, n_s)$. Here, $e_i=\{0, \dots, 1, \dots , 0\}$ (the one is located in the $i$-th component). 
 To find Master equation for $n_1$ we sum over all other components; i.e. 
\begin{eqnarray}
  &&\sum_{n_2=0}^{\infty}\dots \sum_{n_s=0}^{\infty}  \partial_tP(n_1,n_2\dots,n_S) = \\ \nonumber &&=  \sum_{n_2=0}^{\infty}\dots \sum_{n_s=0}^{\infty} \left\{\sum_{i}\left\{ q_{i}^+(\vec{n}-\vec{e_i})P(\vec{n}-\vec{e_i})+q^-_{i}(\vec{n_i}+\vec{e_i}) P(\vec{n}+\vec{e_i})-\left[q_{i}^+(\vec{n})+q_{i}(\vec{n}) \right]P(\vec{n}) \right\}\right\} 
  \end{eqnarray}
  thus
  \begin{equation}
     \partial_t P_1(n_1) = \sum_{n_2=0}^{\infty}\dots \sum_{n_s=0}^{\infty} \left\{\sum_{i}\left\{ q_{i}^+(\vec{n}-\vec{e_i})P(\vec{n}-\vec{e_i})+q^-_{i}(\vec{n_i}+\vec{e_i}) P(\vec{n}+\vec{e_i})-\left[q_{i}^+(\vec{n})+q_{i}(\vec{n}) \right]P(\vec{n}) \right\}\right\} .
  \end{equation}
  We can now use the fact that for every $n_i$:
 \begin{eqnarray}
      \sum_{n_i=0}^{\infty}  q_{i}^+(n_1, \dots n_i-1, \dots, n_S)P(n_1, \dots n_i-1, \dots, n_S) &=& \sum_{n_i=0}^{\infty}  q_{i}^+(n_1, \dots n_i, \dots, n_S)P(n_1, \dots n_i, \dots, n_S)
,  {\rm \ \ and} \\ \nonumber
 \sum_{n_i=0}^{\infty}  q_{n_i}^-(n_1, \dots n_n+1, \dots, n_S)P(n_1, \dots n_i+1, \dots, n_S)&=& \sum_{n_i=0}^{\infty}  q_{n_i}^-(n_1, \dots n_i, \dots, n_S)P(n_1, \dots n_i, \dots, n_S)
 \end{eqnarray}
  [note that   $q_{n_i}^+(n_1, \dots -1, \dots, n_S)P(n_1, \dots, -1, \dots, n_S)=q_{n_i}^-(n_1, \dots 0, \dots, n_S)P(n_1, \dots 0, \dots, n_S)=0$]. Thus, the above equation is given by
\begin{eqnarray}
  \partial_t P_1(n_1) &=& \sum_{n_2=0}^{\infty}\dots \sum_{n_s=0}^{\infty} \left\{ q_{1}^+(\vec{n}-\vec{e_1})P(\vec{n}-\vec{e_1})+q^-_{1}(\vec{n}+\vec{e_1}) P(\vec{n}+\vec{e_1})-\left[q_{1}^+(\vec{n})+q^-_{1}(\vec{n}) \right]P(\vec{n})\right\}.
\end{eqnarray}
For simplicity, we define $F^{+}(\vec{n})\equiv q_1^+(\vec{n})P(\vec{n})$ and $F^{-}(\vec{n})\equiv q_1^-(\vec{n})P(\vec{n})$, thus Eq. 
can be written as 
\begin{eqnarray}
    \partial_t P(n_1) =&& \sum_{n_2=0}^{\infty}\dots \sum_{n_s=0}^{\infty} \left\{ F^{+}(n_1-1,n_2,\dots)-F^{+}(n_1,n_2,\dots)
     +F^{-}(n_1+1,n_2,\dots)-F^{-}(n_1,n_2,\dots) \right\}.
\end{eqnarray}
By using z-transform ($n_1\rightarrow z$), which is defined for a function $k(n_1)$ as $K(z)=\sum_{n_1=0}^{\infty} k(n_1) z^{-n_1} $, we obtain
\begin{eqnarray}
    \partial_t P(z) = \sum_{n_2=0}^{\infty}\dots \sum_{n_s=0}^{\infty} F_{\rm right}(z,n_2,\dots)(1-z^{-1})+F_{\rm left}(z,n_2,n_3\dots)(1-z) 
\end{eqnarray}
\{used ${\cal Z}[g(n)-g(n-1)]=[1-z^{-1}]\hat{G}(z)$, and ${\cal Z}[g(n+1)-g(n)]]=[1-z]\hat{G}(z)-zg(0)$ \}. Stationary solution; $\partial_t P(z)=0 $ and re-organize the equation yields 
\begin{eqnarray}
     \sum_{n_2=0}^{\infty}\dots \sum_{n_s=0}^{\infty}F_{\rm right}(z,n_2,n_3,\dots)=\sum_{n_2=0}^{\infty}\dots \sum_{n_s=0}^{\infty} F_{\rm left}(z,n_2,n_3,\dots)\frac{1-z}{z^{-1}-1}=\sum_{n_2=0}^{\infty}\dots \sum_{n_s=0}^{\infty}F_{\rm left}(z,n_2,n_3\dots)z .
\end{eqnarray}
Then, we use the inverse z-transform ($z\rightarrow n_1$), and find
\begin{eqnarray}
      \sum_{n_2=0}^{\infty}\dots \sum_{n_s=0}^{\infty}q^+_{1}(\vec{n})P(\vec{n})= \sum_{n_2=0}^{\infty}\dots \sum_{n_s=0}^{\infty}q^-_{1}(\vec{n}+\vec{e_1})P(\vec{n}+\vec{e_1}). 
\end{eqnarray}
We use Bayes formula; $P(n_1,n_2,n_3,\dots,n_S)=P(n_2,n_3,\dots, n_S|n_1)P(n_1)$ and obtain
\begin{eqnarray}
     \langle q_{n_1}^+(\vec{n})|n_1\rangle_{n_2,n_3,\dots,n_S}P_1(n_1) = \langle q_{n_1}^-(\vec{n}+\vec{e_1}) |n_1+1\rangle_{n_2,\dots n_S} P_1(n_1+1), 
\end{eqnarray}
where $ \langle *|n_1\rangle_{n_2, \dots, n_S}\equiv\langle *|n_1\rangle \equiv\sum_{n_2=0}^{\infty}\dots \sum_{n_S}^{\infty} (*)P(n_2, \dots, n_S|n_1) $. 
Up to now there are no assumption in the derivation, and the above is general. For our case, as specified in the main text,  $\langle q_{1}^+(\vec{n})|n_1\rangle=\mu + r^+ n_1$ (note that $q^+_{i}$ depends solely on $n_i$) and $\langle q_{1}^-(\vec{n})|n_1\rangle=n_1\left(r^-+r n_1/K + r \rho \sum_{j\neq 1 }\langle n_j |n_1\rangle /K \right)$.  Additionally, from symmetry,  $P_i(n_i)=P_j(n_j) = P(n)$ for every $i,j$.
Thus, solving the recursive equation and obtain
\begin{eqnarray}
    P(n)&=&P(0)\prod_{n'=1}^{n}\frac{q^{+}(n'-1)}{\langle q^-(\vec{n})|n'\rangle}= \label{eq:exact_appendix}
    \\ \nonumber
    &=&P(0)\prod_{n'=1}^{n}\frac{r^+(n'+a)}{n\left(r^-+r n'/K + r\rho \sum_{j\neq 1 }\langle n_j |n'\rangle /K \right)}= P(0)\frac{(r^+)^{n}(a)_{n}}{n!\prod_{n'=1}^{n}\left(r^-+r n'/K + r\rho \sum_{j\neq 1 }\langle n_j |n_1\rangle /K \right)}.
\end{eqnarray}
where
 $a=\mu/r^+$ and $(a)_{n} \equiv a(a+1)\dots (a+n-1)$, in the Pochhammer symbol.  Here, $P(0)$ is given from normalization.
We emphasize that the above abundance distribution $P(n)$ in \eqref{eq:exact_appendix} is exact, means no approximations have been taken so far.

Note, that the denominator in the exact solution depends on the effect of interactions from all other species over $n_1$,  through the term $\sum_{j\neq 1}\langle n_j |n_1 \rangle $. Therefore, in order to provide an explicit expression to $P_1(n_1)$, we need to use some approximations. Three approximation approaches, and a discussion about their limitations are given in the following subsections. We note that all the presented approximations below provide decent results. However, we find that  none of them manage to provide adequate fit for every set of parameters, see Figures.  

\subsection*{Approximation Approach I: Estimating $\sum_{j\neq i}\langle n_j|n_i\rangle $ Using Mean-Field Approximation }
We assume $\langle n_j |n_i \rangle = \langle n_j \rangle $. Thus, 
\begin{equation}
    P(n)\approx P(0)\frac{(r^+)^{n}(a)_{n}}{n!\prod_{n'=1}^{n}\left(r^-+r n'/K + r\rho \sum_{j\neq 1 }\langle n_j\rangle /K \right)} = P(0)\frac{(r^+)^{n}(a)_{n}}{n!\prod_{n'=1}^{n}\left(r^-+r n'/K + r\rho (S-1)\langle n\rangle /K \right)},
    \label{eq:MF1}
\end{equation}
where the last equality is given from symmetry;  $\langle n_j \rangle = \langle n_i \rangle = \langle n\rangle $ for every species $i,j$. In addition, by definition, $\langle n \rangle = \sum_{n=0}^{\infty }n P(n)$, hence
\begin{equation}
    \langle n \rangle \approx P(0)\sum_{n=0}^{\infty}n\frac{(r^+)^{n}(a)_{n}}{n!\prod_{n'=1}^{n}\left(r^-+r n'/K + r\rho (S-1)\langle n\rangle /K \right)}
    \label{eq:MF1_closure}
\end{equation}
where $P(0)=1/{_1}F_1[a,b;c]$ is the normalization coefficient,  with ${_1}F_1[a,b;c]$ is the Kummer confluent hypergeometric function, $a=\frac{\mu}{r^+}$, $b=\frac{r^- K + r \rho (S-1) \langle n \rangle }{r}+1$ and $c=\frac{r^+ K}{r}$. Solving numerically the above implicit equation and evaluate $\langle n\rangle$. The last step is to substitute the numerical solution of $\langle n\rangle$, obtained from \eqref{eq:MF1_closure}, into \eqref{eq:MF1}.    

\subsection*{Approximation Approach II: Estimating $\langle J|n\rangle $ using Convolution }
The exact solution can be written as 
\begin{eqnarray}
    P_1(n_1)=P(n)= P(0)\frac{(r^+)^{n}(a)_{n}}{n!\prod_{n'=1}^{n}\left(r^-+r (1-\rho)n'/K + r\rho \langle J |n_1\rangle /K \right)},
\end{eqnarray}
where $J=\sum_{i=1}^{\infty} n_i $ is the total population size. Here we assume that the total number of individuals in the system, $J$, is weekly depends on $n_1$. Thus $J$ is an independent random variable. Hence, 
\begin{equation} 
    P_1(n_1)=P(n_1|\langle J |n_1 \rangle ) \approx P(n_1|J)=P(0)
    \frac{(a)_{n_1} \Tilde{c}^{n_1}}{n_1 ! (\Tilde{b}+1)_{n_1} } 
\end{equation}
with $a=\frac{\mu}{r^+}$, $\tilde{b}= \frac{r^-K+r\rho J}{r(1-\rho)}$, and $\tilde{c}=\frac{r^+ K}{r(1-\rho)}$ [note that both $\tilde{b}$ and $\tilde{c}$ differ from $b$ and $c$ defined in previous subsection].
Moreover, we assume that the species levels are mutually independent, means ${\cal P}(n_1,\dots n_S) \approx \prod_i P_i(n_i)$. Thus, the PDF of $\sum_i n_i$ reads 
\begin{equation}
    P\left(\left.\sum_i n_i\right|J\right)=\underbrace{P_1(n_1|J)*P_2(n_2|J)* \dots * P_S(n_S|J)}_{S {\rm \ times}}
\end{equation}
where $A*B$ means the convolution of $A$ with $B$. $P\left(\sum_i n_i|J\right)$ is the `analytical' PDF to have $\sum_i n_i$ individuals where we assume that a single species PDF is $P_1(n_1|J)$ with a given $J$. 
To capture the fact that $J$ has a meaning of number of individuals as well, we consider
\begin{equation}
    P(J)\approx \frac{{\rm Prob}\left(\left.\sum_i n_1=J\right|J\right)}{\sum_J {\rm Prob}\left(\left.\sum_i n_1=J\right|J\right)},
\end{equation}
where $P(J)$ is the approximated distribution of $J$. 
Then
\begin{equation}
    P_1(n_1) = \sum_{J}P_1(n_1|J) P(J)
\end{equation}
is the approximated PDF. 

Note that when $S$ is large, we find
\begin{equation}
    P\left(\left.\sum_i n_i\right|J\right) \sim  {\cal N}\left(S\langle n_1 |J \rangle, S \cdot Var(n_i) \right),
\end{equation}
thus $P(J)\approx {\rm Prob}(\sum_i n_i =J|J)$ reaches its maximum in the vicinity of $J$ which satisfies $J\approx S \langle n_i |J \rangle = \left\langle \sum_i n_i |J \right\rangle $. Furthermore, for the approximation $P(J)\approx {\rm Prob}(\sum_i n_i =J|J)$, the values of $J$ where $J\ll S\langle n_i |J \rangle  $ or $J\gg S\langle n_i |J \rangle $ are highly improbable, due to the Gaussian nature of $P(\sum_i n_i|J)$ for large $S$.
 
\subsection*{Approximation Approach III: Estimating $\langle J|n\rangle $ using Mean-Field Approximation } In a similar fashion to previous approximation approaches, we assume $\langle J|n\rangle \approx \langle J\rangle  $, thus
\begin{eqnarray}
    P(n) \approx P(0)\frac{(r^+)^{n}(a)_{n}}{n!\prod_{n'=1}^{n}\left(r^-+r (1-\rho)n'/K + r\rho \langle J \rangle /K \right)}=  P(0)\frac{(r^+)^{n}(a)_{n}}{n!\prod_{n'=1}^{n}\left(r^-+r (1-\rho)n'/K + r\rho S\langle n\rangle  /K \right)}.
\end{eqnarray}
Then, $\langle n \rangle$ is given by the numerical solution of
\begin{eqnarray}
    \langle n \rangle = \sum_{n=0}^{\infty} n P(0)\frac{(r^+)^{n}(a)_{n}}{n!\prod_{n'=1}^{n}\left(r^-+r (1-\rho)n'/K + r\rho S\langle n\rangle  /K \right)}.
\end{eqnarray}
where here the normalization factor is $P(0)=1/{_1}F_1[a,b;c]$ with $a=\frac{\mu}{r^+}$, $\tilde{\tilde{b}}= \frac{r^-K+r\rho \langle J\rangle }{r(1-\rho)}$, and $\tilde{c}=\frac{r^+ K}{r(1-\rho)}$.

\subsection*{Interspecies Correlations and Limitations of the Approximations Approaches}

\section*{Derivations of Boundary Equations}

\subsection*{Boundaries for Richness Regimes} For the boundaries defined from richness, we use
$    \langle S^* \rangle = S \left(1-P(0)\right) 
$, where $P(0)$ obtained numerically from the approximated SAD. Note that in the mean-filed approximations $P(0)$ is explicitly given as Kummer confluent hypergeometric function. Then, the transition between full richness to partial coexistence is given with $S P(0) = 1/2$ (as the arithmetic mean between the two boundaries). Similarly, the transition boundary between partial coexistence and excluded regime is drawn where   $SP(0)=S-3/2$.  

\subsection*{Derivation of $\tilde{n}$ } The boundaries defined by the modalities can be given directly through the modalities obtained from the above approximations, see Figures. However, we have found that using the mean-field approximation allows us to derived a closed expression for the boundaries.     

The transition between neutral-like to bimodal regimes, is defined by the presence  or absence of a local maximum in a real positive level. In another words, in the neutral-like regime $P(n)>P(n+1)$, since the SAD is monotony decreasing, while in the bimodal regime, there is $n>0$ where $P(n)<P(n+1)$. Thus the boundary between the regimes occurs where
  \begin{eqnarray}
      P(n)=P(n+1).
    \end{eqnarray}
Using the $P(n)$ from the mean-field approximation, we find
    \begin{eqnarray}
      P(0)\frac{(a)_n c^n}{n! (b+1)_n }&=&P(0) \frac{(a)_{n+1} c^{n+1}}{(n+1)! (b+1)_{n+1} } 
      \\
      \frac{(b+1)_{n+1}(n+1)!}{n! (b+1)_n }&=& \frac{(a)_{n+1} c^{n+1}}{(a)_n c^n } 
      \\
      n(b+n)&=& c (a+n-1)
      \\
      n&=& \frac{(c-b)\pm \sqrt{(c-b)^2+4(a-1)c}}{2}
  \end{eqnarray}
Substitute $a=\frac{\mu}{r^+}$, $b=\frac{r^- K + r \rho (S-1) \langle n \rangle }{r}+1$ and $c=\frac{r^+ K}{r}$ yields
\begin{eqnarray}
   \tilde{n}\approx
    \frac{K-\rho(S-1)\langle n\rangle}{2}\left\{1 \pm \sqrt{1+4\frac{(\mu-r^+) K}{r(K-\rho(S-1)\langle n\rangle)^2}}\right\}.
    \end{eqnarray}

   
\subsection*{Boundaries for Modalities Regimes}
In the neutral-like regime the SAD is monotonously decreasing. Therefore, the neutral-like regime is sufficiently define by either imaginary $\tilde{n}$, or real but negative $\tilde{n}$. The transition line between real and imaginary $\tilde{n}$, i.e. where $\Im(\tilde{n})=0$, is given by  
\begin{eqnarray}
    {r\left[K-\rho  (S-1)\langle n\rangle\right]^2}=4{(r^+-\mu) K}.
\end{eqnarray}
The transition line between the negative to positive $\tilde{n}$, where $\tilde{n}=0$, is drawn where
\begin{equation}
    \frac{(K-\rho(S-1)  \langle n\rangle)^2}{4}=1+ \frac{4K(\mu-r^+)}{r(K-\rho(S-1)  \langle n\rangle)^2}.
\end{equation}
Therefore, the neutral-like regime is defines as the union of the regions defined by both two equations above.

The second boundary we derive is the border of uni-modality region with positive probable abundance. At zero abundance, the abundance distribution poses either local maximum ($P(0)>P(1)$; correspond to bimodality or neutral-like regimes) or local minimum ($P(0)<P(1)$; unimodality). Consequently, the boundary between these two possibilities is given by $P(0)=P(1)$.  Thus, following the zero-flux equation, we obtain
\begin{equation}
    q^+(0) = \langle q^-(\vec{n})|1\rangle  \Longrightarrow  \mu = r^- +\frac{r}{K} \left[1-\rho + \rho \langle J\rangle\right]. 
\end{equation}
 





\section*{Dependence on Carrying Capacity $K$}



\section*{Simulation Details}
All simulations are generated using Gillespie algorithm, with $10^7$ time steps. 


\section*{Kinetics}

As was mentioned in the main text, we concentrate on time scales associated with mean-first-passage-time (MFPT) from some initial abundance $i$, to a final one $f$. This MFPT, denoted as $\langle T(i\rightarrow f)\rangle$, is inversely proportional to the transition rate from $i$ to $j$, where the first-passage times are exponentially distributed (see Figures). 
In a unidimensional process in an interval $[0,\infty)$, the MFPTs we consider are  
\begin{eqnarray}
      \langle T (\tilde{x} \rightarrow 0) \rangle =   \sum_{y=0}^{\tilde{x}-1}\frac{1}{q^+(y)P(y)}\sum_{z=y+1}^{\infty}{P(z)}, & &
  \langle T (0 \rightarrow 0) \rangle =  \frac{1}{q^+(0)P(0)}=  \frac{1}{\mu P(0)}, \\ \nonumber 
  \langle T (0 \rightarrow \tilde{x}) \rangle =  \sum_{y=0}^{\tilde{x}}\frac{1}{q^+(y)P(y)}\sum_{z=0}^{y}{P(z)}, & &
  \langle T (\tilde{x} \rightarrow \tilde{x}) \rangle =    \frac{1}{P(\tilde{x})[q^+(\tilde{x})+q^-(\tilde{x})]} .
\end{eqnarray}
These expressions are exact for processes in one dimension \cite{gardiner1985handbook}, i.e. for a single species were no other species affect its dynamics.  Here we have found, using simulation, that these MFPTs agree with the multi-dimensional scenario, where many species evolve (see Figures). Hence, we substitute the SAD obtained from simulation,  as $P(n)$ in the expressions above and estimated the rates with $ R(a \rightarrow b)=1/\langle T(a \rightarrow b)\rangle $. A similar approach is used to estimate the other rate ratio. 

\section*{Fokker-Planck formulation}

In the analysis so far we have used $n_i$ to represent a discrete variable, however we may approximate the master equation in a continuum should some large characteristic system size exist.
Assuming here that $K >> 1$ is the system size, we define $x_i$ to be the corresponding continuous limit of $n_i$.
The variable may be rescaled by the characteristic system size, $y_i = x_i / K$
This continuous approximation has as probability density $ P(\vec{n},t) = p(\vec{y},t)/K$ and we may write the the scaled rates as $q_i^{+/-}(\vec{n})=K Q_i^{+/-}(\vec{y})$ where
\begin{align*}
    Q_i^{+}(\vec{y}) &= r^+ y_i + \mu / K \\
    Q_i^{+}(\vec{y}) &= r^- y_i + r y_i \left( y_i + \sum_{j\neq i}\rho_{j,i} y_j \right)
\end{align*}
is defined on the continuum.
Then we can write the corresponding master equation as
\begin{equation}
    \label{eq:master-eq-cont}
    \partial_t  p(\vec{x};t) =
    K \sum_{i}\left\{Q^+_i(\vec{y}-\vec{e}_i)p(\vec{y}-\vec{e}_i,t)+ 
    Q^-_i (\vec{y}+\vec{e}_i)p(\vec{y}+\vec{e}_i,t) - \left[Q^+_i(\vec{y})+Q^-_i(\vec{y})\right]p(\vec{y},t)\vphantom{\vec{e}_i} \right\}
\end{equation}
wherein $\vec{e}_i$ is the change in abundance $\vec{y}$ from the respective event, which in our single birth-death process will be $\vec{e}_i=(\delta_{1i}/K, \delta_{2i}/K, ..., \delta_{Si}/K)$, in other words a vector of zeros except for $1/K$ located at species $i$.
To go from the master equation to the Fokker-Planck equation, we Taylor expand each of the expressions from the right-hand side of \ref{eq:master-eq-cont}.
As such,
\begin{multline*}
    Q^+_i(\vec{y}-\vec{e}_i)p(\vec{y}-\vec{e}_i,t)= Q^+_i(\vec{y})p(\vec{y},t)+\sum_j (-(\vec{e}_i)_j) \frac{\partial}{\partial y_j}\left( Q^+_i(\vec{y})p(\vec{y},t)\vphantom{\frac{1}{1}}\right)
    + \frac{1}{2!}\sum_j \sum_k (\vec{e}_i)_j(\vec{e}_i)_k \frac{\partial^2}{\partial y_j \partial y_k}\left( Q^+_i(\vec{y})p(\vec{y},t)\vphantom{\frac{1}{1}}\right)+...
\end{multline*}
\begin{multline*}
    Q^-_i(\vec{y}+\vec{e}_i)p(\vec{y}+\vec{e}_i,t)= Q^-_i(\vec{y})p(\vec{y},t)+\sum_j (\vec{e}_i)_j \frac{\partial}{\partial x_j}\left( Q^-_i(\vec{y})p(\vec{y},t)\vphantom{\frac{1}{1}}\right)
    + \frac{1}{2!}\sum_j \sum_k (\vec{e}_i)_j (\vec{e}_i)_k \frac{\partial^2}{\partial y_j \partial y_k}\left( Q^-_i(\vec{y})P(\vec{y},t)\vphantom{\frac{1}{1}}\right)+...
\end{multline*}
Note that $(\vec{e}_i)_j=\delta_{ij}/K$ in our single birth-death event process, which simplifies these equations considerably. 
Now, replacing these expressions in our master equation \ref{eq:master-eq-cont}, we note that we can write our equation in orders of $1/K$.
Thus, we obtain
\begin{multline}
\label{eq:fokker-planck}
    \partial_t  p(\vec{y};t) =
    -\sum_j  \frac{\partial}{\partial y_j}\left[\left( Q^+_i(\vec{y})-Q^-_i(\vec{y})\right)p(\vec{y},t)\vphantom{\frac{1}{1}}\right]
    + \frac{1}{2 K}\sum_j  \frac{\partial^2}{\partial y_j \partial y_k}\left[\left( Q^+_i(\vec{y})+Q^-_i(\vec{y})\right)p(\vec{y},t)\vphantom{\frac{1}{1}}\right] + \mathcal{O}(1/K^2).
\end{multline}
Up to order $1/K$, \ref{eq:fokker-planck} is the Fokker-Planck Equation (FPE) of the process.
Using Ito's prescription for SDEs, this corresponds to the Langevin equation
\begin{equation}
    \label{eq:langevin}
    d y_i = \left( Q^+_i(\vec{y})-Q^-_i(\vec{y}) \right)dt + \sqrt{ \frac{Q^+_i(\vec{y})+Q^-_i(\vec{y})}{K} } dW_i
\end{equation}
where $W_i$ is a standard Wiener process. By multiplying both sides of this equation by the characteristic size $K dy_i = dx_i$,
\begin{equation}
    \label{eq:langevin-LV}
    d x_i = \left( q^+_i(\vec{x})-q^-_i(\vec{x}) \right)dt + \sqrt{ K \left( q^+_i(\vec{x})+q^-_i(\vec{x})\right) } dW_i
\end{equation}
where the force term in the Langevin equation recovers the Lotka-Voltera equation.

\section*{Langevin equation numerical results}
Interestingly, the Langevin equation can also be simulated to see how well the Fokker-Planck approximation does in approximating the distribution. What we find, using Euler method ,  is that in this formulation the complete phase space of the regimes is not reproduced. The `neutral-like' regime extends into lower immigration regimes. Need to elaborate on this some more.

%%% Each figure should be on its own page
\begin{figure}
    \includegraphics[width=0.75\columnwidth]{figures/supp-MFPT.pdf}
    \\
\includegraphics[trim=130 230 130 230, width=0.24\columnwidth]{figures/MFPT_rho1_mu03162new.pdf} 
\includegraphics[trim=130 230 130 230, width=0.24\columnwidth]{figures/MFPT_rho1_mu1_new.pdf} 
\includegraphics[trim=130 230 130 230, width=0.24\columnwidth]{figures/MFPT_rho01_mu03162.pdf} 
\\
\includegraphics[trim=130 230 130 230, width=0.24\columnwidth]{figures/MFPT_rho1_mu03162_new.pdf}  
\includegraphics[trim=130 230 130 230, width=0.24\columnwidth]{figures/MFPT_rho1_mu1_new_part2.pdf}
\includegraphics[trim=130 230 130 230, width=0.24\columnwidth]{figures/MFPT_rho01_mu03162_part2.pdf}
\caption{Turnover time statistics}
\end{figure}

\begin{figure}
    \centering
    \includegraphics{}
    \caption{Varying carrying capacity}
    \label{fig:varyK}
\end{figure}





\begin{figure}
    \centering
    \includegraphics[width=0.75\columnwidth]{figures/supp-langevin.pdf}
    \caption{Langevin numerical simulations. left: master equation noise, right: mehta noise.}
    \label{fig:langevin}
\end{figure}

\begin{table}\centering
\caption{This is a table}

\begin{tabular}{lrrr}
Species & CBS & CV & G3 \\
\midrule
1. Acetaldehyde & 0.0 & 0.0 & 0.0 \\
2. Vinyl alcohol & 9.1 & 9.6 & 13.5 \\
3. Hydroxyethylidene & 50.8 & 51.2 & 54.0\\
\bottomrule
\end{tabular}
\end{table}

%%% Add this line AFTER all your figures and tables
\FloatBarrier

\movie{Type legend for the movie here.}

\movie{Type legend for the other movie here. Adding longer text to show what happens, to decide on alignment and/or indentations.}

\movie{A third movie, just for kicks.}

\dataset{dataset_one.txt}{Type or paste legend here.}

\dataset{dataset_two.txt}{Type or paste legend here. Adding longer text to show what happens, to decide on alignment and/or indentations for multi-line or paragraph captions.}

\bibliography{bibliography}


\end{document}
